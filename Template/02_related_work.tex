\chapter{Related Work}

Here you should summarize academic work that is related to the problem you are trying to solve. There are a few key points to this section:
\begin{enumerate}
    \item You need to properly cite works here, like this \cite{confer_2005_activity}.
    \item Your references should all be in the BibTeX format inside a .bib file.  You can do one of the following:
    \begin{itemize}
        \item Paste in the BibTeX citation of whatever work you are referring to (usually available in places like IEEE Xplore as "Cite This")
        \item Use a reference manager like Zotero or Mendeley, which allows exporting a .bib file
        \item Manually create the BibTeX entry, as shown in \cite{bibtexEntries}
    \end{itemize}
    \item Introduce each work by using the format shown (plus the citation):
    \begin{itemize}
        \item 1 author: "Confer \cite{confer_2005_activity} ... "
        \item 2 authors: "Eshelman and Schaffer \cite{eshelman93real} ..."
        \item 3 or more authors: "Kapsalisand et al. \cite{kapsalis93solving} ... "
    \end{itemize}
    \item At most, you should spend a paragraph on each work in question.  Consult your advisor for further specifications, but you should summarize:
    \begin{itemize}
        \item What is the core approach/idea
        \item What are its relevant results (if applicable)
        \item If possible, how the work is deficient (in particular if you are intending to address that deficiency)
    \end{itemize}
    \item \textbf{Your references should NOT be mostly composed of web links and/or online articles (e.g., Medium).}  Your Master's project/thesis is an academic work and should contain references to other peer-reviewed academic papers/journals/articles.
    \item Your references should be \textbf{recent, within 2 or at most 3 years}.  Some works you will cite will be seminal works that date further back, but in general you should be covering the state of the art.
    \item \textbf{Make sure you have sufficient references discussed here.}  It is not acceptable to discuss, say, 2 papers and think it is satisfactory coverage of the state of the art.
\end{enumerate}

\section{The Foobar Methods}

It is perfectly reasonable and encouraged to break up groups of methods into different subsections.  If you do so, introduce what is the unifying aspect of these approaches (e.g., "Some approach the problem by using the Foobar method").

\section{The Other Methods}

Here is where you would talk about these methods.

At the very end of this chapter, it is common to summarize what is overall missing in previous work that you specifically will address in your approach.